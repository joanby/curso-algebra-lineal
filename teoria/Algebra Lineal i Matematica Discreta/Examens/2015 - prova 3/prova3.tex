\documentclass[12p,spanish]{article}
\usepackage[spanish]{babel}
\usepackage[ansinew]{inputenc}
\usepackage[T1]{fontenc}
\usepackage{graphicx}
\usepackage{multicol}
\usepackage{longtable}
\usepackage{array}
\usepackage{multirow}
\usepackage{geometry}                		
\geometry{letterpaper}                   		
\usepackage{graphicx}
\usepackage{amssymb}
\usepackage{color}


\setlength{\textwidth}{16cm}
\setlength{\textheight}{24cm}
\setlength{\oddsidemargin}{-0.3cm}
\setlength{\topmargin}{-1.3cm}


\newcommand{\sC}{{\cal C}}
\newcommand{\sF}{{\cal F}}
\newcommand{\sL}{{\cal L}}
\newcommand{\sU}{{\cal U}}
\newcommand{\sX}{{\cal X}}
\newcommand{\eop}{{\Box}}

\newcommand{\ar}{A^{(r)}}
\newcommand{\HH}{{\bf H}}
\newcommand{\sS}{{\cal S}}
\newcommand{\Img}{\mbox{Img}}

\def\N{I\!\!N}
\def\R{I\!\!R}
\def\Z{Z\!\!\!Z}
\def\Q{O\!\!\!\!Q}
\def\C{C\!\!\!\!I}


\newcount\problemes
\problemes=0

\def\probl{\advance\problemes by 1
\vskip 2ex\noindent{\bf \the\problemes \hbox{ } }}

\graphicspath{ {im/} }


\newcommand{\notimplies}{%
  \mathrel{{\ooalign{\hidewidth$\not\phantom{=}$\hidewidth\cr$\implies$}}}}





\begin{document}
\pagestyle{empty}

\parindent =0 pt
{\bf Algebra Lineal. Primer de Telem�tica. 
\hfill Primer parcial - 26 de Gener, 2016}

\vspace{1.5 cm}
\probl Donats els vectors $\vec u = (2,0,0), \vec v = (0,1,-3)$ i $\vec w = a\vec u + b\vec v$, quina condici� han de complir els escalars $a$ i $b$ per tal de que:
\begin{enumerate}
\item $\vec w$ sigui ortogonal al vector $(1,1,1)$.
\item $\vec w$ sigui unitari.
\item $\vec w$ sigui paral�lel al vector $(1,2,-6)$.
\item Per $a=1, b=-1$, calculau el vector de longitud 3 en sentit oposat a $\vec w$.
\end{enumerate}
\vspace{0.6 cm}

\probl Considerau els vectors del conjunt $C=\{(1,1,1),(1,1,2),(1,2,3)\}$.
\begin{enumerate}
\item Demostrau que formen una base de $\mathbb R^3$.
\item Trobau respecte d'aquesta base les coordenades del vector $(5,1,-3)$.
\item Trobau respecte de la base can�nica les coordenades del vector que t� per coordenades $(3,1,-1)$ en la base $C$.  
\end{enumerate}
\vspace{2.5 cm}

{\bf Algebra Lineal. Primer de Telem�tica. 
\hfill Segon parcial - 26 de Gener, 2016}
 

\probl Un endomorfisme $f$ de $\mathbb{R}^3$ est� determinat per $f(x,y,z)=(2y+z, x-4y, 3x)$ en la base can�nica. Es demana
\begin{enumerate}
\item Trobar el nucli i la imatge de $f$.
\item Trobar la matriu de $f$ en aquesta base
\item Trobar la matriu de $f$ en la base $V$ constituida pels vectors $v_1=(1,1,1), v_2=(1,1,0), v_3=(1,0,0)$
\item L'expressi� anal�tica de $f$ en aquesta base $V$
\end{enumerate}
\vspace{0.6 cm}

\probl Donada la matriu
\[A=\left(\begin{array}{ccc}a+1 & a-1 & a \\a-1 & a+1 & a \\0 & 0 & 1\end{array}\right)\]
\begin{enumerate}
\item Trobau els valors propis de la matriu $A$ en funci� del par�metre $a$.
\item En funci� dels valors propis trobats, estudiau si $A$ �s o no �s diagonalitzable segons els valors del par�metre $a$ (no cal trobar expl�citament els vectors propis).
\item Per $a=0$, calculau els vectors propis de A. Digau en aquest cas qui �s $D$ i $VP$.
\end{enumerate}

\vspace{2.5 cm}


{\bf Algebra Lineal. Primer de Telem�tica. 
\hfill Tercer parcial - 26 de Gener, 2016}

\probl Donada la funci� booleana $f(x,y,z)=\overline{(\bar{x}y)}(\bar{x}+xy\bar{z})$
\begin{enumerate}
\item Donau la forma can�nica disjuntiva
\item Donau la taula de veritat de la funci�
\item Dibuixau el seu mapa de Carnaugh i emprau-lo per donar-ne una simplificaci�. 
\end{enumerate}
\vspace{0.6 cm}



\probl Considerem el joc del $N$-d�mino generalitzat. En aquest joc es tenen fitxes, cadascuna de les quals est� etiquetada amb dos nombres, $a, b \in \{0, . . . , N\}$, i que indicarem per $[a \cdot b]$; les fitxes es poden girar; es a dir, la fitxa $[a \cdot b]$ i la fitxa $[b \cdot a]$ s�n indistingibles. 
\begin{enumerate}
\item Suposant que no hi ha fitxes repetides, amb quantes fitxes juguem al $3$-d�mino? I al $4$-d�mino? I al $5$-d�mino?\item Suposant que no hi ha fitxes repetides, amb quantes fitxes juguem al $N$-d�mino?
\item Una partida de d�mino �s una seq��ncia de fitxes, $[a_1 \cdot b_1][a_2 \cdot b_2] \cdots [a_k \cdot b_k]$, de manera que els nombres adjacents de fitxes diferents coincideixen, �s a dir, amb $b_i = a_{i+1}$ per a tot $i = 1,\cdots,k-1$. �s possible fer una partida on s'emprin totes les fitxes del joc? Justifica la teva resposta emprant la terminolog�a vista a classe. 
\end{enumerate}

\vspace{0.6 cm}
\end{document}  