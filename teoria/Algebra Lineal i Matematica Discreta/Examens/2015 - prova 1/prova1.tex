\documentclass[12p,spanish]{article}
\usepackage[spanish]{babel}
\usepackage[ansinew]{inputenc}
\usepackage[T1]{fontenc}
\usepackage{graphicx}
\usepackage{multicol}
\usepackage{longtable}
\usepackage{array}
\usepackage{multirow}
\usepackage{geometry}                		
\geometry{letterpaper}                   		
\usepackage{graphicx}
\usepackage{amssymb}
\usepackage{color}


\setlength{\textwidth}{16cm}
\setlength{\textheight}{24cm}
\setlength{\oddsidemargin}{-0.3cm}
\setlength{\topmargin}{-1.3cm}


\newcommand{\sC}{{\cal C}}
\newcommand{\sF}{{\cal F}}
\newcommand{\sL}{{\cal L}}
\newcommand{\sU}{{\cal U}}
\newcommand{\sX}{{\cal X}}
\newcommand{\eop}{{\Box}}

\newcommand{\ar}{A^{(r)}}
\newcommand{\HH}{{\bf H}}
\newcommand{\sS}{{\cal S}}
\newcommand{\Img}{\mbox{Img}}

\def\N{I\!\!N}
\def\R{I\!\!R}
\def\Z{Z\!\!\!Z}
\def\Q{O\!\!\!\!Q}
\def\C{C\!\!\!\!I}


\newcount\problemes
\problemes=0

\def\probl{\advance\problemes by 1
\vskip 2ex\noindent{\bf \the\problemes \hbox{ } }}

\graphicspath{ {im/} }


\newcommand{\notimplies}{%
  \mathrel{{\ooalign{\hidewidth$\not\phantom{=}$\hidewidth\cr$\implies$}}}}





\begin{document}
\pagestyle{empty}

\parindent =0 pt
{\bf Algebra Lineal. Primer de Telem�tica. 
\hfill Primer parcial - 13 de Novembre, 2015}
 

\vspace{0.6 cm}
\probl 
\begin{enumerate}
\item (0.5 pts) Enunciau la condici� necess�ria i suficient per tal que una matriu tingui inversa. 
\item (0.5 pts) Determinau el valor de $a$ pel qual la matriu
\[A = \left(\begin{array}{ccc}a & 0 & 3 \\0 & -1 & 0 \\0 & 0 & -1\end{array}\right)\]
�s invertible. 
\item (1 pt) Calculau la inversa per $a=-1$.
\end{enumerate}
\vspace{0.6 cm}
\probl (2 pts) Calculau l'angle que formen els vectors $\vec a$ i $\vec b$ sabent que $||\vec a|| = 3$, $||\vec b|| = 5$ i $||\vec a + \vec b|| = 7$.
\vspace{0.6 cm}
\probl (2.5 pts) Justifica quins dels seg�ents conjunts de l'espai vectorial donat s�n subespais vectorials. D'aquells que, en efecte siguin subespais vectorials, d�na una base i la dimensi� dels mateixos. Justifica totes les respostes. 
\begin{enumerate}
\item Els vectors de  $\mathbb R^n$ els coeficients dels quals s�n nombres enters (�s a dir de $\mathbb Z$).
\item Els vectors del pla $\mathbb R^2$ situats sobre l'eix $OX$.
\item Els vectors del pla $\mathbb R^2$ situats b� sobre l'eix $OX$ o b� sobre l'eix $OY$.
\item Els vectors de $\mathbb R^n$ tals que la seva primera i darrera coordenades s�n iguals.
\item Els vectors de $\mathbb R^n$ tal que les seves coordenades senars s�n 0.
\item Els vectors de $\mathbb R^n$ tal que totes les seves coordenades senars s�n iguals.
\item Els vectors del pla $\mathbb R^2$  que tenen el seu origen i l'extrem sobre una recta qualssevol del pla.
\item Els vectors de l'espai $\mathbb R^3$ els extrems dels quals no es troben sobre una recta donada.
\item Els vectors del pla amb els extrems sobre el primer quadrant.
\item Les matrius sim�triques amb coeficients reals dins l'espai $M_{n\times n} (\mathbb R)$
\end{enumerate}
\vspace{0.6 cm}
\probl Sigui $E$ un $\mathbb R$-espai vectorial de dimensi� 4 amb base $B=\{u_1, u_2, u_3, u_4\}$. Es defineixen els vectors
\[v_1 = 2 u_1+u_2-u_3\ \ \ v_2 = 2u_1+u_3+2u_4\ \ \ v_3 = u_1+u_2-u_3\ \ \ v_4=-u_1+2u_3+3u_4\]
\begin{enumerate}
\item (0.75 pt) Demostrau que $B'=\{v_1,v_2,v_3,v_4\}$ �s una base de $E$.
\item (1.5 pt) Trobau les coordenades del vectors $u_1,u_2,u_3,u_4$ en la base $B'$.
\item (1.25 pt) El vector $\vec x \in \mathbb R^4$ t� coordenades $(1,2,0,1)$ en la base $B$. Calcula'n les coordenades en la base $B'$. 
\end{enumerate}
\vspace{0.6 cm}

\textbf{Temps m�xim per fer la prova: 3 hores. }

\vspace{0.6 cm}
\end{document}  