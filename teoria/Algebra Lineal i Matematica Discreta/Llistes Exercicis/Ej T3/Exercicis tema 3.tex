\documentclass[12p,spanish]{article}
\usepackage[spanish]{babel}
\usepackage[ansinew]{inputenc}
\usepackage[T1]{fontenc}
\usepackage{graphicx}
\usepackage{multicol}
\usepackage{longtable}
\usepackage{array}
\usepackage{multirow}
\usepackage{geometry}                		
\geometry{letterpaper}                   		
\usepackage{graphicx}
\usepackage{amssymb}
\usepackage{color}


\setlength{\textwidth}{16cm}
\setlength{\textheight}{24cm}
\setlength{\oddsidemargin}{-0.3cm}
\setlength{\topmargin}{-1.3cm}


\newcommand{\sC}{{\cal C}}
\newcommand{\sF}{{\cal F}}
\newcommand{\sL}{{\cal L}}
\newcommand{\sU}{{\cal U}}
\newcommand{\sX}{{\cal X}}
\newcommand{\eop}{{\Box}}

\newcommand{\ar}{A^{(r)}}
\newcommand{\HH}{{\bf H}}
\newcommand{\sS}{{\cal S}}
\newcommand{\Img}{\mbox{Img}}

\def\N{I\!\!N}
\def\R{I\!\!R}
\def\Z{Z\!\!\!Z}
\def\Q{O\!\!\!\!Q}
\def\C{C\!\!\!\!I}


\newcount\problemes
\problemes=0

\def\probl{\advance\problemes by 1
\vskip 2ex\noindent{\bf \the\problemes \hbox{ } }}

\graphicspath{ {im/} }


\newcommand{\notimplies}{%
  \mathrel{{\ooalign{\hidewidth$\not\phantom{=}$\hidewidth\cr$\implies$}}}}




\begin{document}

\pagestyle{empty}

\parindent =0 pt
{\bf Problemes d'Algebra Lineal. Primer de Telem�tica. 
\hfill Tema 3 - Espais Vectorials}

\vspace{0.6 cm}
\probl Considereu els vectors $\vec u = (3,0,0), \vec v = (1,0,-3)$ i $\vec w = (0,-2,4)$ 
\begin{enumerate}
\item Estudiau si s�n linealment dependents o linealment independents. 
\item Trobau, si �s possible, una combinaci� lineal d'aquests vectors tal que el seu resultat sigui el vector $(-2,4,1)$. 
\item Finalment, �s possible obtenir qualsevol vector $(x,y,z)\in \mathbb R^3$ com a combinaci� lineal de $\vec u, \vec v, \vec w$? Raonau la resposta.
\end{enumerate}

\vspace{0.4cm}
\probl Donats els vectors $\vec v = (1,2,3), \vec u = (2,4,0), \vec w =(3,-2,3)$.
\begin{enumerate}
\item Troba tres nombres $a,b,c$ tals que $a \vec v + b \vec u + c \vec w = \vec 0$.
\item �s $\vec v $ combinaci� lineal de $\vec u$ i $\vec w$?
\item �s $\vec w $ combinaci� lineal de $\vec u$ i $\vec v$?
\item Expressa el vector $(6,0,9)$ com a combinaci� lineal dels tres vectors. 
\end{enumerate}

\vspace{0.4cm}
\probl Determina si els seg�ents conjunts de vectors de $\mathbb R^3$ s�n linealment dependents o independents. Estudia el seu rang:
\begin{enumerate}
\item $(1,-2,1),(2,1,-1),(7,-4,1)$.
\item $(1,-3,7),(2,0,-6),(3,-1,-1), (2,4,-5)$.
\item $(1,2,-3),(1,-3,2),(2,-1,5)$.
\item $(2,-3,7),(0,0,0),(3,-1,-4)$.
\item $(1,1,1),(1,-1,5)$.
\item $(1,1,1),(1,2,3),(2,-1,1)$.
\item $(2,2,-1),(4,2,-2),(7,-4,1)$.
\item $(1,0,1,3),(0,0,2,-5),(-1,2,0,7)$.
\end{enumerate}

\vspace{0.4cm}
\probl Demostrau que $\vec a = (x,y)$ i $\vec b = (z,t)$ de $\vec R^2$ s�n linealment dependents si i nom�s si $xt-yz=0$.

\vspace{0.4cm}
\probl Trobau els valors d'$a$ per tal que els vectors $(-2,a,a), (a,-2,a),(a,a,-2)$ siguin linealment dependets.

\vspace{0.4cm}
\probl Donat el conjunt de vectors $U=\{ \vec u_1,\vec u_2,\vec u_3,\vec u_4   \}$ tal que $\vec u_4$ �s combinaci� lineal de $\vec u_1$ i $\vec u_2$, i $\vec u_3$ �s combinaci� lineal de $\vec u_1$ i $\vec u_4$, discutiu si els vectors de $U$ s�n LI o LD, i quin �s el rang de $U$.

\vspace{0.4cm}
\probl Determinar si els vectors $\vec v = (1,0,2),\vec u = (0,1,-3), \vec w = (1,1,0), \vec z = (0,0,-1)$ de l'espai vectorial $\mathbb R^3 $ formen un sistema generador de $\mathbb R^3$. 

\vspace{0.4cm}
\probl Si  $U=\{ \vec u_1,\vec u_2,\vec u_3,\vec u_4 , \vec u_5  \}$ �s un conjunt de vectors de rang 3, comenta la veritat o falsetat de les seg�ents afirmacions. 
\begin{enumerate}
\item Tots els subconjunts de $U$ de tres vectors s�n conjunts lliures.
\item El vector $\vec u_3$ �s linealment independent dels altres.
\item Els vectors del subconjunt $U=\{\vec u_1, \vec u_2, \vec u_3\}$ s�n linealment independents.
\item Algun subconjunt de $U$ de tres vectors �s lliure (els seus vectors s�n linealment independents) ja que el rang �s el nombre m�xim de vectors LI que podem trobar en el conjunt. Per tant �s suficient amb que hi hagi almenys un subconjunt de 3 vectors LI.
\end{enumerate}

\vspace{0.4cm}
\probl Estudiar si els seg�ents conjunts de vectors compleixen els requisits per formar una base de $\mathbb R^3$:
\begin{enumerate}
\item $A=\{(1,1,1),(3,1,-1),(-4,2,8)\}$
\item $B=\{(3,2,1),(2,1,0),(1,0,0)\}$
\end{enumerate}

\vspace{0.4cm}
\probl Considerau els vectors $(1,2,-1),(2,1,-1)$. Trobau un vector tal que amb els altres dos formi una base de $\mathbb R^3$.

\vspace{0.4cm}
\probl Sigui $\vec u = (2,1,3)$ en la base can�nica. Calculau les components de $\vec u$ en la nova base $B=\{(1,2,0),(1,-1,1),(0,1,1)\}$.

\vspace{0.4cm}
\probl Provau que el conjunt $C=\{(1,1,1),(1,1,2),(1,2,3)\}$ formen una base de $\mathbb R^3$. Trobau, respecte d'aquesta base, les coordenades del vector $(5,1,-3)$. Trobau les coordenades en la base can�nica del vector que t� per coordenades $(3,1,-1)$ en la base $C$.

\vspace{0.4cm}
\probl Expressau $\vec v = (1,-2,5)\in \mathbb R^3$ com a combinaci� lineal dels vectors $\vec u_1= (1,-3,2), \vec u_2 = (2,-4,-1), \vec u_3 = (1,-5,7)$. Formen $\vec u_1, \vec u_2, \vec u_3$ una base de $\mathbb R^3$?

\vspace{0.4cm}
\probl Provau que el conjunt $D=\{(1,0,0,-1),(0,1,-1,0),(0,1,0,-1),(0,1,1,1)\}$ formen una base de $\mathbb R^4$. Obteniu les coordenades del vector $(-3,2,1,-2)$ en aquesta base. Si el vector $\vec u$ t� coordenades $(3,0,-1,1)$ en la base $D$ obteniu-ne les coordenades en la base can�nica. 

\vspace{0.4cm}
\probl Sigui $B_1=\{\vec u_1, \vec u_2, \vec u_3\}$ una base d'un espai vectorial $V$. Sabent que
\[\left\{\begin{array}{ccccc}\vec v_1 & = & 2\vec u_1 & - & \vec u_2 \\\vec v_2 & = & -\vec u_1 & + & \vec u_3 \\\vec v_3 & = & \vec u_2 &  & \end{array}\right.\]
obteniu les coordenades de $\vec v_1, \vec v_2, \vec v_3$ en la base $B_1$. Demostrau que $B_2=\{\vec v_1, \vec v_2, \vec v_3\}$ tamb� forma una base de $V$. Siguin $(2,1,3)$ les coordenades del vector $\vec u$ en la base $B_1$, calculau les coordenades en la base $B_2$.

\vspace{0.4cm}
\probl Sigui $\{\vec v_1, \vec v_2, \vec v_3\}$ una base de $V$. Demostrau que si 
\[\left\{\begin{array}{ccccccc}\vec u_1 & = & \vec v_1 &  &  &  &  \\\vec u_2 & = & \vec v_1 & + & \vec v_2 &  &  \\\vec u_3 & = & \vec v_1 & + & \vec v_2 & + & \vec v_3\end{array}\right.\]
el conjunt $\{\vec u_1, \vec u_2, \vec u_3\}$ tamb� �s una base de $V$.

\vspace{0.4cm}
\probl Considerem $\vec e_1, \vec e_2, \vec e_3$ tres vectors LI.
\begin{enumerate}
\item Demostrau que $\vec u_1, \vec u_2, \vec u_3$ s�n tamb� LI, on
\[\left\{\begin{array}{ccccc}\vec u_1 & = & 2\vec e_1 & + & \vec e_3 \\\vec u_2 & = & 3\vec e_3 &  &  \\\vec u_3 & = & \vec e_2 & - & \vec e_3\end{array}\right.\]
\item Donat un vector $\vec v = 2\vec e_1+\vec e_2-3\vec e_3$, expressau-lo com a combinaci� lineal de $\vec u_1, \vec u_2, \vec u_3$.
\end{enumerate}


\vspace{0.4cm}
\probl Donada la base $\{(1,2,0,0), (-1,0,1,1), (0,0,-2,1), (-1,0,-1,0)\}$ de $\mathbb R^4$ i donat $\vec u$  de components $(-3,2,1,-2)$ en aquesta base, obteniu les components en la base formada pels vectors $(1,0,0,-1), (0,1,-1,0), (0,1,0,-1), (0,1,1,1)$.


\vspace{0.4cm}
\probl Sigui $B_V = \{\vec v_1, \vec v_2, \vec v_3\}$ una base d'$\mathbb R^3$ i siguin els vectors 
\[\left\{\begin{array}{ccccccc}\vec u_1 & = & 3\vec v_1 & + & 2\vec v_2 & - & \vec v_3 \\\vec u_2 & = & 4\vec v_1 & + & \vec v_2 & + & \vec v_3 \\\vec u_3 & = & 2\vec v_1 & - & \vec v_2 & + & \vec v_3\end{array}\right.\]
Provau que:
\begin{enumerate}
\item $B_U = \{\vec u_1, \vec u_2, \vec u_3\}$ �s una base de $\mathbb R^3$.
\item Trobau les coordenades dels vectors $\vec v_1, \vec v_2, \vec v_3$ en la base $B_U$.
\item El vector $\vec x\in \mathbb R^3$ t� coordenades $(1,2,3)$ en la base $B_V$. Calcula les coordenades de $\vec x$ en la base $B_U$.
\end{enumerate}

\vspace{0.4cm}
\probl Expressau el vector $(3,1,4)$ en la base d'$\mathbb R^3$ formada pels vectors $(1,-2,-1), (1,-1,0),(0,0,3)$.

\vspace{0.4cm}
\probl Sigui la base $B_1 = \{(1,0,0,-1),(0,1,-1,0),(0,1,0,-1),(0,1,1,1)\}$ de $\mathbb R^4$. Obteniu  les components en aquesta base del vector $\vec u$ que t� coordenades $(-3,2,1,-2)$ en la base $B_2$ formada pels vectors $(1,2,0,0), (-1,0,1,1),(0,0,-2,1), (-1,0,-1,0)$.

\vspace{0.4cm}
\probl Indicau quins dels seg�ents conjunts s�n subespais vectorials dels espais indicats. Donau una base en el cas que siguin subespais:
\begin{enumerate}
\item $U=\{(x,y,z)\in \mathbb R^3\ :\ x-z = 0 \}$
\item $V=\{(x,y,z)\in \mathbb R^3\ :\ x-y+z = 0 \}$
\item $W=\{(x,y,z)\in \mathbb R^3\ :\ x=0,\ y+z = 0 \}$
\item $X=\{(a,b,0)\in \mathbb R^3\}$
\item $Y=\{(a,b,c,d)\in \mathbb R^4\ :\ b=a+c+d+1\}$
\item $Z=\{(a,b,c,d)\in \mathbb R^4\ :\ b=a+c,\ d=2a \}$
\end{enumerate}

\vspace{0.4cm}
\probl Trobau una base del subespai vectorial $E$ generat per $\vec v_1 = (1,-2,0,3), \vec v_2 = (2,-5,-3,6), \vec v_3 = (0,1,3,0), \vec v_4 = (2,-1,4,-7)$ i $\vec v_5 = (5,-8,1,2)$.

\vspace{0.4cm}
\probl Trobau una base del subespai vectorial $F=\langle (1,2,1,0), (0,0,1,0), (1,2,0,0) \rangle$

\vspace{0.4cm}
\probl Provau que
\[\langle (1,-1,-1,1), (1,-2,-2,1), (0,1,1,0) \rangle = \langle (1,0,0,1), (0,-1,-1,0) \rangle\]
Donau una base d'aquest subespai vectorial. 

\vspace{0.4cm}
\probl Sigui $E$ l'espai vectorial generat pels vectors $\vec v_1, \vec v_2, \vec v_3$, $E=\langle \vec v_1, \vec v_2, \vec v_3 \rangle$ on $\vec v_1 = (2,1,0,3), \vec v_2 = (3,-1,5,2), \vec v_3 = (-1,0,2,1)$. Determinau la dimensi� de $E$ i dir si els vectors $(1,1,1,1)$ i $(2,3,-7,3)$ pertanyen o no a $E$.

\vspace{0.4cm}
\probl Demostrau que el conjunt de les matrius de la forma
\[\left(\begin{array}{cc}x-3y & 5y \\-4y & x+3y\end{array}\right)\]
amb $x,y\in \mathbb Q$ formen un subespai vectorial de l'espai de matrius d'ordre 2 sobre el cos dels racionals $M_2(\mathbb Q)$.

\end{document}  