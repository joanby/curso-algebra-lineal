\documentclass[12p,spanish]{article}
\usepackage[spanish]{babel}
\usepackage[ansinew]{inputenc}
\usepackage[T1]{fontenc}
\usepackage{graphicx}
\usepackage{multicol}
\usepackage{longtable}
\usepackage{array}
\usepackage{multirow}
\usepackage{geometry}                		
\geometry{letterpaper}                   		
\usepackage{graphicx}
\usepackage{amssymb}
\usepackage{color}


\setlength{\textwidth}{16cm}
\setlength{\textheight}{24cm}
\setlength{\oddsidemargin}{-0.3cm}
\setlength{\topmargin}{-1.3cm}


\newcommand{\sC}{{\cal C}}
\newcommand{\sF}{{\cal F}}
\newcommand{\sL}{{\cal L}}
\newcommand{\sU}{{\cal U}}
\newcommand{\sX}{{\cal X}}
\newcommand{\eop}{{\Box}}

\newcommand{\ar}{A^{(r)}}
\newcommand{\HH}{{\bf H}}
\newcommand{\sS}{{\cal S}}
\newcommand{\Img}{\mbox{Img}}

\def\N{I\!\!N}
\def\R{I\!\!R}
\def\Z{Z\!\!\!Z}
\def\Q{O\!\!\!\!Q}
\def\C{C\!\!\!\!I}


\newcount\problemes
\problemes=0

\def\probl{\advance\problemes by 1
\vskip 2ex\noindent{\bf \the\problemes \hbox{ } }}

\graphicspath{ {im/} }


\newcommand{\notimplies}{%
  \mathrel{{\ooalign{\hidewidth$\not\phantom{=}$\hidewidth\cr$\implies$}}}}





\begin{document}
\pagestyle{empty}

\parindent =0 pt
{\bf Problemes d'Algebra Lineal. Primer de Telem�tica. 
\hfill Tema 5 - Diagonalitzaci�}

\vspace{0.6 cm}
\probl Calculau el polinomi caracter�stic i els valors propis de la matriu
\[R = \left(\begin{array}{cccc}1 & 0 & 0 & 1 \\2 & -1 & 0 & -1 \\0 & -1 & 2 & 0 \\-1 & 0 & 0 & -2\end{array}\right)\]
Calculau tamb� les multiplicitats algebraiques. 

\vspace{0.4 cm}
\probl Calculau els vectors propis de la matriu de l'exercici anterior aix� com les multiplicitats geom�triques dels mateixos. 

\vspace{0.4 cm}
\probl Calcular els vectors propis, els valors propis i les multiplicitats algebraiques i geom�triques de les seg�ents matirus
\begin{multicols}{2}
\begin{enumerate}
\item \[A = \left(\begin{array}{cc}0 &-1 \\1&0\end{array}\right)\]
\item \[B  = \left(\begin{array}{ccc}2 & -4 & -4 \\0 & -2 & 0  \\-4 & 4 & 2 \end{array}\right)\]
\item \[C  = \left(\begin{array}{ccc}-5 & 3 & 3 \\-2 & 4 & 2 \\-7 & 3 & 5 \end{array}\right)\]
\item \[D = \left(\begin{array}{cccc}2 & -1 & 3 & -3 \\0 & -1 & 0 & 0 \\-1 & 3 & 1 & 0 \\-1 & 2 & 2 & 0\end{array}\right)\]
\end{enumerate}
\end{multicols}

\vspace{0.4 cm}
\probl Dir quines de les seg�ents matrius s�n diagonalitzables
\begin{enumerate}
\item \[A = \left(\begin{array}{cc}1 &4 \\0&3\end{array}\right)\]
\item \[B  = \left(\begin{array}{ccc}1 & 2 & 0 \\2 & -1 & 5  \\0 & 5 & 1 \end{array}\right)\]
\item \[C  = \left(\begin{array}{ccc}1 & 1 & 3 \\0 & 2 & 0  \\-2 & 0 & 3 \end{array}\right)\]
\end{enumerate}
Diagonalitza en els casos en que sigui possible.

\vspace{0.4 cm}
\probl Calcula el polinomi caracter�stic i valors propis de la matriu:
 \[A = \left(\begin{array}{cccc}1 & 0 & 0 & 1 \\2 & -1 & 0 & -1 \\0 & -1 & 2 & 0 \\-1 & 0 & 0 & -2\end{array}\right)\]
Diagonalitza si �s possible.



\vspace{0.4 cm}
\probl Sigui la matriu:
 \[A =\left(\begin{array}{ccc}1 & -1 & -1 \\-1 & 1 & -1  \\-1 & -1 & 1 \end{array}\right)\]
Troba el seus valors propis, vectors propis, determina els subespais propis associats. Diagonalitza la matriu $A$ si �s possible.


\vspace{0.4 cm}
\probl Sigui la matriu:
 \[A =\left(\begin{array}{ccc}0 & 0 & 4 \\1 & 2 & 1  \\2 & 4 & -2 \end{array}\right)\]
Troba el seus valors propis, vectors propis. �s diagonalitzable?


\vspace{0.4 cm}
\probl Sigui la matriu:
 \[A =\left(\begin{array}{ccc}3 & -1 & 1 \\0 & 2 & 0  \\1 & -1 & 3 \end{array}\right)\]
Trobar si �s possible una base de $\mathbb R^3 $ formada per vectors propis de $A$.


\vspace{0.4 cm}
\probl Estudia segons el valor del par�metre $a$ si la matriu:
 \[A =\left(\begin{array}{ccc}1 & -4 & 0 \\0 & 4a & 0  \\0 & 0 & 3 \end{array}\right)\]
�s o no �s diagonalitzable.



\vspace{0.4 cm}
\probl Estudia segons el valor del par�metre $a$ si la matriu:
 \[B =\left(\begin{array}{ccc}a & 2 & 0 \\0 & -1 & 0  \\0 & 0 & 1 \end{array}\right)\]
�s o no �s diagonalitzable.



\vspace{0.4 cm}
\probl Estudia segons el valor dels par�metres $a$ i $b$ si la matriu:
 \[C =\left(\begin{array}{ccc}0 & 1 & b \\a^2 & 0 & 0  \\0 & 0 & 1 \end{array}\right)\]
�s o no �s diagonalitzable.



\vspace{0.4 cm}
\probl Donada la matriu
 \[A =\left(\begin{array}{ccc}a+1 & a-1 & a \\a-1 & a+1 & a  \\0 & 0 & 1 \end{array}\right)\]

\begin{enumerate}
\item Estudiar si $A$ �s o no �s diagonalitzable segons els valors del par�metre $a$.
\item Pel cas de $a=0$ calcular $A^n$.
\end{enumerate}

\end{document}  