\documentclass[12pt,t]{beamer}
% \documentclass[t]{beamer}
\usepackage[utf8]{inputenc}
\usepackage[catalan]{babel}
\usepackage{verbatim}
\usepackage{hyperref}
\usepackage{amsfonts,amssymb,amsmath,amsthm,wasysym}
\usepackage{listings}
\usepackage[T1]{fontenc}        
\usepackage{accents}

\usepackage{multicol} % indice en 2 columnas
\usepackage{centernot}
\usepackage{amsmath}% http://ctan.org/pkg/amsmath

\usepackage{color}


\usepackage{graphicx}
\graphicspath{ {im/} }


\newcommand{\notimplies}{%
  \mathrel{{\ooalign{\hidewidth$\not\phantom{=}$\hidewidth\cr$\implies$}}}}


%\usetheme{default}
\usetheme[hideothersubsections,left]{Marburg}
\usecolortheme{sidebartab}
\useinnertheme[shadow]{rounded}
\setbeamertemplate{navigation symbols}{}
% \useoutertheme[footline=empty,subsection=true,compress]{infolines}
 %\useoutertheme[footline=empty,subsection=true,compress]{miniframes}
\usefonttheme{serif}

\setbeamertemplate{caption}[numbered]
%\usepackage{pgfpages}
%\pgfpagesuselayout{4 on 1}[a4paper,border shrink=5mm,landscape]
%\setbeamertemplate{footline}[frame number]

\newcommand{\red}[1]{\textcolor{red}{#1}}
\newcommand{\green}[1]{\textcolor{green}{#1}}
\newcommand{\blue}[1]{\textcolor{blue}{#1}}
\newcommand{\gray}[1]{\textcolor{gray}{#1}}
\renewcommand{\emph}[1]{{\color{red}#1}}

\setbeamertemplate{frametitle}
{\begin{centering}
\medskip
\color{blue}
\textbf{\insertframetitle}
\medskip
\end{centering}
}
\usecolortheme{rose}
\usecolortheme{dolphin}
\mode<presentation>

\newcommand{\CC}{\mathbb{C}}
\newcommand{\RR}{\mathbb{R}}
\newcommand{\ZZ}{\mathbb{Z}}
\newcommand{\NN}{\mathbb{N}}
\newcommand{\QQ}{\mathbb{Q}}
\renewcommand{\leq}{\leqslant}
\renewcommand{\geq}{\geqslant}

\theoremstyle{plain}
\newtheorem{teorema}{Teorema}
\newtheorem{formula}[theorem]{Fórmula}
\theoremstyle{definition}
\newtheorem{pregunta}{Pregunta}
\newtheorem{definicio}{Definició}

\title[\red{22350 - Álgebra Lineal y Matemática Discreta}]{}

\author[]{Juan Gabriel Gomila}


\date{}


\begin{document}
\beamertemplatedotitem


\lstset{backgroundcolor=\color{green!50}}
\lstset{breaklines=true}
\lstset{basicstyle=\ttfamily}


\begin{frame}
\vfill
\begin{center}
\gray{\Huge 22350 - Álgebra Lineal y Matemática Discreta}\\[3ex]
\Large Curs 2018-19\\
Grau en Enginyeria Telemàtica\\
UIB
\end{center}
\vfill
\end{frame}

\section{Presentació}

\subsection{Coordenades}


\begin{frame}
\frametitle{Qui som?}
\vspace*{-0.2cm}
\centering
\begin{minipage}[c]{0.45\linewidth}
\small \begin{center}
\red{Grups grans:}\\
  Juan Gabriel Gomila\\
\medskip

%\includegraphics[width=0.7\linewidth]{cescM.jpg}\\
\includegraphics[height=0.7\linewidth]{jb.jpg}\\
\medskip

\red{Despatx} d'associats \\
(al costat dels ascensors del hall)\\
 Ed. Anselm Turmeda
\medskip

\red{Tutories:} Avisau\medskip

\red{Mail:} juangabriel.gomila@uib.es
\end{center}
\end{minipage}



\end{frame}


\subsection{Objectius}
\begin{frame}
\frametitle{Objectius}
\medskip

Aprendre les tècniques d'àlgebra lineal i matemàtica discreta des d'un punt de vista molt aplicat, continuant la vostra formació  en base als vostres coneixements de batxillerat:
\medskip

\begin{itemize}
\item Matrius, sistemes, determinants i vectors.
\item Espais vectorials, bases, aplicacions lineals i diagonalització
\item Optimització i programació lineal
\item Teoria de conjunts, àlgebra de Boole i teoria de grafs 

\end{itemize}

\end{frame}

\subsection{Activitats}
\begin{frame}
\frametitle{Activitats}

\red{Classes en grup gran:}
\smallskip

Teoria en forma de transparències per part meva. Resolució de problemes per part vostra.\bigskip
\pause

\red{Classes en grup mitjà:}
\smallskip

Resolució de problemes individuals o en equips petits
\bigskip
\pause

\red{Feina a casa} ($\sim$ 4--5 hores setmanals):
\smallskip

Estudiar, resoldre problemes per entregar, participar activament en el fòrum plantejant i resolent dubtes dels vostres companys,...

\end{frame}


\subsection{Avaluació}

\begin{frame}
\frametitle{Avaluació}
\medskip

\red{1. Examens:} 3 examens, consistents en algunes qüestions, exercicis i problemes dels continguts vists a classe
\bigskip

\begin{itemize}
\item El \emph{primer examen parcial} (Dilluns 22 d'Octubre) pesa un \blue{$30\%$} a la nota final.
\medskip
\item El \emph{segon examen parcial} (Dilluns 3 de Desembre) pesa un \blue{$30\%$} a la nota final.
\medskip

\item El \emph{tercer examen final} (Dilluns 14 de Gener) pesa un \blue{$30\%$} a la nota final.
\end{itemize}
\bigskip

\end{frame}


\begin{frame}
\frametitle{Avaluació}
\medskip

\red{1. Controls:} Important!
\bigskip

\begin{itemize}
\item L'examen parcial allibera matèria per a l'examen final.
\item Per tal de ser avaluat positivament del global de l'assignatura, els estudiants hauran d'obtenir una puntuació mínima de 4 punts (sobre 10) en el global agregat dels exàmens.
\item El día de l'Examen Final, es farà una recuperació dels parcials I i II. 
\item En cas d'anar a la recuperació de Febrer, s'haurà de recuperar tota l'assignatura i no es guardaran les notes dels controls realitzats. La nota final serà l'obtinguda en aquesta convocatoria

\medskip
\end{itemize}
\end{frame}




\begin{frame}
\frametitle{Avaluació}
\bigskip

\red{2. Seminari i Taller de problemes} La nota mitjana pesa un \blue{$10\%$} a la nota final.
\bigskip

\red{3. Problemes per fer a casa}: No té pes a l'avaluació final però es molt recomanable per dur el curs de manera fluïda. 

\red{4. Participació activa al fòrum}: Es valorarà amb dècimes extra (fins a un màxim d'un punt) a la nota final la vostra participació de forma activa al fòrum, tant per plantejar dubtes de forma oberta, com per resoldre les dels vostres companys. 

\end{frame}





\begin{frame}
\frametitle{Avaluació}

$$
\begin{array}{rl}
\mbox{\red{\bf Nota final}} & =0.30\cdot \mbox{\blue{Primer control}}\\
& \quad+0.30 \cdot\mbox{\blue{Segon control}}\\
& \quad+0.30 \cdot\mbox{\blue{Tercer control}}\\
& \quad +0.10 \cdot\mbox{\blue{Exercicis de taller i problemes}}\\
& \quad +\mbox{\blue{Dècimes extra per participació}}
\end{array}
$$
\bigskip

Per la recuperació febrer, s'hauran de recuperar els tres controls. Cap altra activitat d'avaluació (exercicis de taller, o participació) serà recuperable.


\end{frame}


\subsection{Bibliografia}

\begin{frame}
\frametitle{Bibliografia}

\begin{itemize} 

\item Apunts de classe disponibles en Campus Extens 
\item Rosen, Kenneth H. Matemática discreta y sus aplicaciones 5a ed. McGraw-Hill, 2004 
\item Ferrer, M. Pilar; Lerís, M. Dolores; Ribera, J. Manual sobre álgebra lineal. Prensas Universitarias de Zaragoza, 2003, ISBN: 84-7733-672-5

\end{itemize}

\end{frame}



\begin{frame}
\frametitle{Bibliografia}

\begin{itemize} 


\item  García Merayo, Félix. Matemática discreta 2a ed. Thomson-Paraninfo, 2005 
\item J.R. Evans, E. Minieka. Optimization algorithms for networks and graphs. Second edition. 
\item Grimaldi, Ralph P. Matemáticas discreta y combinatoria:introducción y aplicaciones. Addison-Wesley Iberoamericana, 1997. 
\item S. Pemmaraju, S. Skiena. Computational Discrete Mathematics. Cambridge.
\end{itemize}

\end{frame}



\subsection{Consells finals}


\begin{frame}
\frametitle{Consells finals}

\begin{itemize}

\item Procurau accedir a Campus Extens de forma regular per consultar qualsevol esdeveniment, exercici o notícia sobre l'assignatura.

\item Posau-vos com abans millor una foto de perfil (vostra no de qualsevol cosa que vos pugui agradar\ldots) a Campus Extens per així­ coneixer-vos. 

\item Veniu a classe de teoria amb les transparències impreses o amb soport digital (tablet, ordinador\ldots)

\end{itemize}
\end{frame}

\begin{frame}
\frametitle{Consells finals}

\begin{itemize}

\item Consultau el \emph{Tauler d'Anuncis de Campus Extens}, serà la manera bàsica de comunicar-nos amb vosaltres per temes de logí­stica.

\item Procurau fer totes les tasques que us encarregui, i com abans millor. Un $10\%$ de l'assignatura us pot salvar en un moment donat... Amb un 4.5 de mitjana de parcials i un 5 de les tasques aprovaríeu\ldots

\item Utilitzau el fòrum individualitzat per tutories per contactar amb nosaltres.

\end{itemize}

\end{frame}


\begin{frame}
\frametitle{Consells finals}

No dediqueu més temps del necessari a l'assignatura:
\begin{itemize}
\item No repetiu exercicis innecessàriament.
\medskip

\item El exercicis estan pensats per ser resolts en menys d'una hora, si trigau més, demanau ajuda.
\medskip
\item Les tutories són més barates que les academies (de fet són de franc). Aprofitau-les des del primer dia. 

\item Sense estudiar tot costa molt \ldots
\medskip

\item Emprau els recursos de Campus Extens.
\medskip
\end{itemize}

\end{frame}

\begin{frame}
\frametitle{Consells finals}

\begin{itemize}

\item El frau en qualsevol de les activitats d'avaluació serà durament penalitzat segons l'Article 32 del Reglament Acadèmic de la UIB.

\item Si en les entregues de problemes, detectam entregues de dos o més grups amb clars indicis de còpia mútua o de font comú, els grups en qüestió seran penalitzats.

\end{itemize}

\end{frame}




\begin{frame}
\frametitle{Consells finals}

\begin{itemize}

\item Anau a les festes de la Univerland, Biofestes i gaudiu de la vostra vida universitària, que mai més tindreu tantes ocasions  de sortir de festa com estudiants de la UIB ;)

\end{itemize}
\centering

\includegraphics[height=0.6\linewidth]{univerland.jpg}\\

\end{frame}

\end{document}


